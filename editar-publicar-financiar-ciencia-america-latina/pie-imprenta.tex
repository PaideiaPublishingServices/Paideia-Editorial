% pie-imprenta.tex
\thispagestyle{empty}

% Página de créditos y catalogación
\vspace*{2cm}

\begin{center}
\textbf{\Large Editar, publicar y financiar ciencia en América Latina}\\[0.5em]
\textit{\large Perspectivas, experiencias y distopías en las dinámicas de la comunicación científica en la región}\\[2em]
\end{center}

% Cuadro de catalogación
\noindent\textbf{Catalogación en fuente}\\[0.5em]

\begin{small}
\begin{tabular}{|p{12cm}|}
\hline
\textbf{Del Rio Riande, Gimena (coord.) ; Pantaleo, Patricio Iván (coord.)} \\
\quad Editar, publicar y financiar ciencia en América Latina: perspectivas, experiencias y distopías en las dinámicas de la comunicación científica en la región / Gimena del Rio Riande y Patricio Iván Pantaleo, coordinadores. - [Edición a completar] - Córdoba: Paideia Editorial, 2025. \\
\quad Libro digital, PDF ; HTML ; ePUB \\[0.5em]
\quad Archivo Digital: descarga y online \\
\quad ISBN PDF [A COMPLETAR] \\
\quad ISBN HTML [A COMPLETAR] \\
\quad ISBN ePUB [A COMPLETAR] \\[0.5em]
\quad 1. Comunicación Científica. 2. Edición Académica. 3. Acceso Abierto. 4. América Latina. I. Del Rio Riande, Gimena, coord. II. Pantaleo, Patricio Iván, coord. III. Título. \\
\quad CDD 070.5 \\
\hline
\end{tabular}
\end{small}

\vspace{1cm}

% Datos editoriales
\noindent\textbf{Edición:} [A COMPLETAR]\\
\textbf{ISBN PDF:} [A COMPLETAR]\\
\textbf{ISBN HTML:} [A COMPLETAR]\\
\textbf{ISBN ePUB:} [A COMPLETAR]\\
\textbf{DOI:} 10.62059/editorial.l001\\
\textbf{Hecho el depósito que marca la Ley 11.723}\\[1em]

% Información editorial
\noindent\textbf{Derechos de primera publicación:} 2025 \quad Paideia Editorial\\
Córdoba - Argentina\\
Web: \url{https://paideiaeditorial.net}\\
Email: contact@paideiastudio.net\\[1em]

% Derechos de autor
\noindent\textbf{Derechos de autor:}\\
Los derechos de cada capítulo pertenecen a sus respectivos autores, así como los de la coordinación general de la obra a Gimena del Rio Riande y Patricio Iván Pantaleo.\\
Los autores solo ceden a Paideia Editorial derechos de primera publicación.\\[1em]

% Coordinadores
\noindent\textbf{Coordinadores:}\\
Gimena del Rio Riande (ORCID: 0000-0002-8997-5415)\\
Patricio Iván Pantaleo (ORCID: 0000-0002-8104-8975)\\[1em]

% Información de formatos disponibles
\noindent\textbf{Formatos disponibles:}\\
PDF (Impresión y lectura digital)\\
HTML (Lectura web interactiva)\\
ePUB (Dispositivos de lectura electrónica)\\[1em]

% Disponibilidad digital
\noindent\textbf{Disponible en:}\\
\url{https://paideiaeditorial.net/libros/editar-financiar-ciencia-latinoamerica/}\\
Repositorio: \url{https://github.com/PaideiaPublishingServices/Paideia-Editorial}\\[1em]

% Logos de licencias y Crossmark
\noindent\textbf{Licencias y certificaciones:}\\[0.5em]
% Creative Commons
\includegraphics[height=1cm]{cc.png}\hspace{0.2cm}\includegraphics[height=1cm]{by.png}\hspace{1cm}%
% Logo Crossmark con enlace funcional
\href{https://crossmark.crossref.org/dialog/?doi=10.62059/editorial.l001&domain=pdf&date_stamp=2025-07-22}{\includegraphics[height=1cm]{crossmark-logo.png}}
\\[0.5em]
Esta obra está bajo una Licencia Creative Commons Atribución 4.0 Internacional.\\
\url{https://creativecommons.org/licenses/by/4.0/}\\
Para verificar política Crossmark de la editorial, visite: \url{https://doi.org/10.62059/paideia.editorial.crossmark}\\[1em]

% Clasificaciones y responsabilidades
\noindent\textbf{Clasificación temática (LEMB):}\\
1. Comunicación en ciencia -- América Latina\\
2. Edición académica\\
3. Acceso libre a la información\\
4. Ciencia abierta\\
5. Política editorial\\[1em]

\noindent\textbf{Nota de responsabilidad:}\\
Las opiniones expresadas en cada capítulo son responsabilidad exclusiva de sus autores y no necesariamente reflejan la posición de Paideia Editorial ni de los coordinadores de la obra.\\[1em]

\vspace*{\fill}

% Pie de página con información de licencia
\begin{center}
\small
\textbf{Obra de acceso abierto distribuida bajo licencia Creative Commons Atribución 4.0 Internacional}\\[0.5em]

\textbf{Permite}\\
Compartir — copiar y redistribuir el material en cualquier medio o formato\\
Adaptar — remezclar, transformar y construir a partir del material\\
para cualquier propósito, incluso comercialmente\\[0.5em]

\textbf{Bajo los siguientes términos}\\
Atribución — Debe otorgar crédito de manera adecuada, brindar un enlace a la licencia, e indicar si se han realizado cambios.\\[1em]

% Información Crossmark
\textbf{Crossmark}\\
Este documento participa en Crossmark, un servicio que proporciona información\\
sobre actualizaciones editoriales después de la publicación.\\
Más información: \url{https://www.crossref.org/crossmark/}
\end{center}

\newpage